\chapter{Virksomhedsanalyse}

ECT A/S er bygget op som et aktieselskab med en bestyrelse til at varetage virksomhedens interesser. Der er tre ejere af firmaet; Henrik, Knud og Ejvind, som ejer hhv. 40, 30 og 30 procent af firmaet. Firmaet blev startet af Henrik og Ejvind, og Knud købte sig ind efter at have været ansat som sælger og markedsføringsmand.

Henrik er den praktiske del af firmaet. Det er ham, der skriver koden, laver printlayout og sågar tegner lidt 3D-prototypetegninger til produkterne. Det er ham, man går til, når forhindringer opstår i arbejdet. Han er uddannet elektronikingeniør på Ingeniørhøjskolen i Århus og har efter endt uddannelse startet firmaet i samarbejde med Ejvind.

Knud har chefposten i firmaet og står for alt det praktiske med godkendelser, kundemøder og kontakt til leverandører. Han er også uddannet elektronikingeniør og har været ansat på B\&O som udviklingsingeniør, da billedrørsfjernsyn var på markedet. Senere har han dog haft meget at gøre med projektledelse i en række andre virksomheder her i blandt Triax A/S og KIRK telecom A/S.

Ejvind har ikke sin daglige gang i firmaet, men han står for at skaffe kunder og investorer til projekter, som ECT A/S kan virkeliggøre. Han er bosiddende i København, men han kommer jævnligt forbi til bestyrelsesmøder i firmaet.

En af ECT A/S's største indkomster har i lang tid været salg af DECT-moduler. DECT er en trådløs kommunikationsstandard brugt i f.eks. trådløse fastnettelefoner. De har også selv udviklet produkter med denne standard.

Derudover laver firmaet opgaver på konsulentbasis for kunder. En kunde kan eksempelvis have et ønske om at lave et produkt, men denne har ikke den tekniske viden herom. Der vil herefter blive lavet en skriftlig aftale, og samarbejdet kan begynde.

\newpage
\section{Produkter}

ECT A/S har været inde over mange produkter. Deres eget produkt Carephone, som er en alarm til ældre mennesker på f.eks. plejehjem, er et produkt, de selv har udviklet med midler fra diverse fonde.

\figur{0.8}{carephone8}{ECT's Carephone}{fig:carephone} 

Derudover har de lavet et utal af produkter for kunder med især DECT-teknologien indbygget.

\figur{0.8}{distribution}{Indkomstfordeling i virksomheden}{fig:distribution}

På figur \ref{fig:distribution} ses at salget af komponenter (DECT-moduler) har været en stor del af ECT A/S's indkomst. Modulerne har været produceret af Spectralink (tidligere: Kirk Telecom), og ECT A/S har haft en handelsaftale om at forhandle modulerne. Pga. omstrukturering og udgåede komponenter til disse har Knud og Henrik dog haft travlt med at finde et alternativ, og det kæmper de stadig lidt med.

\newpage
\section{Vækst}

ECT A/S er i vækst og er kommet ud af 2015 med et pænt overskud. De har en masse bolde i luften og håber på at få nogle opgaver og aftaler i hus, så virksomheden kan vokse i løbet af de næste mange år. 

\figur{0.8}{growth}{Vækstkurve for de seneste år}{fig:growth}