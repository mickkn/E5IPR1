\chapter{Kollegaer og forhold}

Henrik og Knud har taget rigtig godt imod mig. 

Jeg har fået lov til at kører mit eget løb uden en ''chef'' hængende på skulderen hele tiden. Dog har de været gode til lige at høre hvordan det går. Og er meget behjælpelige hvis man hænger fast i et problem eller har nogle spørgsmål. ECT deler bygning med et IT-support firma bestående af én mand, Henrik, som vi holder rundstyksmøde med hver fredag, derfor skiftes man til at komme med rundstykker. Disse møder kan godt trække lidt ud, da det er super hyggeligt lige at få delt røverhistorier ud.

Hos ECT er der altid kaffe på kanden og frokostpauserne styrer man lidt selv. Det er en lille virksomhed så der er ikke faste tidspunkter man spiser.

Man er adskilt i to lokaler, hvor forlokalet er hvor Knud og Henrik sidder og som udgør et mødelokalet. Så er der baglokalet hvor praktikanterne holder til og hvor der er et lille bord med loddekolbe og komponenter. Derudover er der et te-køkken og nogle mindre lagerrum fyldt med sager.

Knud er ikke bleg for at fortælle historier fra de varme lande og om hans oplevelser i sine tidligere jobs og Henrik har altid lige et projekt som han er i gang med derhjemme, for tiden er han ved at bygge en CNC-fræser hjemme i garagen.

