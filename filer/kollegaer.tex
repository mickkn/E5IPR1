\chapter{Kollegaer og forhold}

Henrik og Knud har taget rigtig godt imod mig. 

Jeg har fået lov til at køre mit eget løb uden en ''chef'' hængende på skulderen hele tiden. Dog har de været gode til indimellem at høre, hvordan det går. De har været meget behjælpelige, hvis jeg har hængt fast i et problem eller har haft nogle spørgsmål. ECT A/S deler bygning med et IT-support firma bestående af én mand, Henrik, som vi holder rundstykke-møde med hver fredag, derfor skiftes man til at komme med rundstykker. Disse møder har det med at trække lidt ud, da det altid er hyggeligt lige at få delt røverhistorier ud.

Hos ECT A/S er der altid kaffe på kanden, og frokostpauserne styrer man til dels selv. Det er en lille virksomhed, som har et meget behageligt og fleksibelt arbejdsmiljø.

ECT A/S er delt op i to lokaler. I forlokalet sidder Knud og Henrik og arbejder, og dette lokale udgør også mødelokalet. I baglokalet holder praktikanten til, og her er et arbejdsbord med loddekolbe og komponenter. Derudover er der et te-køkken og nogle mindre lagerrum fyldt med sager.

Knud er ikke en mand, der er bleg for at fortælle historier fra de varme lande og om hans oplevelser i sine tidligere jobs, og Henrik har altid et eller andet projekt, som han er i gang med derhjemme - for tiden er han ved at bygge en CNC-fræser hjemme i garagen. De er begge gode til at skabe en god stemning på arbejdspladsen, og derudover er de yderst kompetente til deres arbejde og samtidig nogle imødekommende og gode rådgivere og kollegaer, som jeg med det samme havde en god kemi med.  