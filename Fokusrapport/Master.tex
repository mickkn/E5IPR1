\input{preamble_ida14}
\raggedbottom
\begin{document}

\tableofcontents*

\chapter{Uddannelsesplan}

I forbindelse med praktikforløbet i ECT A/S kommer der til at være mange forskellige arbejdsopgaver. Firmaet, som består af to mand, har mange projekter i ilden, som betyder, at deres praktikanter får lov at stå og sørge for en masse delopgaver. 

I de 20 uger, jeg skal være i praktik, skal der arbejdes med et Brystpumpe-projekt, som består af en hospitalsversion, der kører på en fast strømkilde og en consumerversion, der kører på batteri. Selve brystpumpe-projektet er som sådan afsluttet af tidligere praktikkanter, men da pumpen er til test, vil der helt sikkert opstå nogle ændringer, som jeg skal være en del af. Der skal f.eks. skiftes fra en 12 V motor til 5 V pga. for stor varmeudvikling i 12 V-versionen.

Derudover har firmaet udviklet et måleapparat til måling af muskelaktivitet på heste. Dette produkt er også blevet afsluttet og kører lige nu test i fabrikken, der skal producere sensorer; dette er jeg også tovholder på.

ECT A/S er specialister i DECT\footnote{Digital Enhanced Cordless Telecommunications} teknologien. Med denne teknologi har de udviklet deres eget produkt CarePhone, som er en ''trådløs telefon'' til plejehjemsbeboere. Produktet sælges ved forskellige forhandlere i Norden.

\figur{0.5}{carephone8.png}{Carephone}{fig:carephone}

Til dette er der dog opstået et ønske om induktiv opladning, så den fysiske forbindelse dermed fjernes. Derfor skal der også bruges en trådløs mulighed for firmware-opdateringer, f.eks. via Bluetooth. Dette er et forskningsprojekt, jeg også får lov at være med til og muligvis også stå for.

\chapter{Fokuspunkter}

\section*{Hvordan anvendes de første semestres teoriundervisning i arbejdet i virksomheden?}

Der bliver mange forskellige opgaver i form af diagramtegning og udregning, C-kodning af mikroprocessorer samt seriel kommunikation. Således er anvendelsen af undervisningen på skolen en kæmpe del af praktikforløbet.

\section*{Hvordan kan erfaring med projektarbejde anvendes?}

Når der skal laves noget ved ECT, skal alt dokumenteres. Mine erfaringer fra projektarbejdet på skolen, hvor dokumentation er en stor del af projektet kan dermed anvendes i, høj grad. Ligeledes skal der også samarbejdes med kollegaer og samarbejdspartnere, som f.eks. fabrikkerne hvor produkterne produceres.

\section*{Overvejelse om fremtidige arbejdsplads, samme type som praktikstedet?}

ECT er en meget lille virksomhed og dvs. at hvis jeg skulle lave noget lignende, når jeg er færdig på ingeniørhøjskolen, ville det betyde, at jeg skulle blive selvstændig og starte et firma op selv. Jeg kunne godt se mig selv som selvstændig, hvis idéen skulle poppe op. 

Jeg ville nok ellers finde mig bedst tilrette i en mellemstor virksomhed, hvor man kunne arbejde intenst med en delopgave i et større udviklingsprojekt. Men det er spændende at få lov til at blive en del af så lille et firma, hvor man som praktikkant får lov til at stå for ''rigtige'' opgaver, i stedet for at blive gemt væk i et hjørne for at lave forskning af noget, som måske aldrig bliver realiseret.

\section*{Hvordan kan erfaringer fra praktikopholdet anvendes ved valg af Bachelorprojekt?}

Der vil helt sikkert komme opgaver og opnåede erfaringer i praktikforløbet, som kan påvirke mit valg af bachelorprojekt; for mig er det er dog for tidligt at sige noget om lige pt.

\section*{Hvordan kan erfaringer fra praktikopholdet anvendes ved valg af tilvalgskurser?}

I ECT får jeg som sagt lov til at lave en bred vifte af projekter/opgaver. Dette vil helt sikkert påvirke min interesse for fag, som kunne være spændende at udvikle kompetencer indenfor, og derfor vil det blive nemmere at kunne vælge de rigtige tilvalgskurser på 6. og 7. semester.

\end{document}
